\documentclass[12pt]{article}

\usepackage{amsmath}

\usepackage{graphicx}

\usepackage{hyperref}

\usepackage[utf8]{inputenc}

\title{Machine Learning with a Fractal Genome: working title}

\author{Ethan Speakman}

\date{2023–08–27}

\begin{document}

\maketitle

\section{Motivation \& Introduction}
\paragraph*{}
The Spliceosome has one of the most important functions of the genome: the removal of introns from exons. 
This critical step takes pre-mRNA and assembles exons for translation into proteins. 
A large ribonuclaticprotein (RNP) structure, the Spliceosome is composed of several small nuclear RNPs (snRNPs) which in turn bind to specific locations on the intron, fold the intron on itself, and splice it out, leaving only the exons\cite{will2011spliceosome}\cite{matera2014day}.
This step occurs shortly after transcription by the Ribosome, 
Following this processes the per-mRNA is further matured and modified before being sent out of the nucleous for translation.
\paragraph*{}
Identification is made even more complicated by the role of alternative splicing.
Alternative splicing allows for a more diverse proteome as the roughly 20,000 genes of the human genome can be expressed in multiple ways: in humans exons can be skipped in different patterns, allowing a single gene have multiple expression forms.
In fact, any gene that has more then 2 exons can be alternatively expressed.
This means that the Spliceosome needs not only to properly identify introns for removal, it also must identify which exons are to be skipped if any.
There are many factors that play into this, such as RNA Secondary Structures and splice site identification, but the precise combination of those factors is not yet fully understood.
\paragraph*{}
For this reason a high fidelity of precise intron and exon identification is required for the Spliceosome to properly excise the introns: failure to do so can have pathological consquences.

And yet errors happens rarely: Duchenne muscular dystrophy, a genetic disease caused by the deletion and skipping of exons for example, effects less then 10 in every 100,000 males, and less then 1 per million in females.
\section{The Role of the Spliceosome}
\paragraph*{}
\section{Background}
\section{Models}
\paragraph*{1D Time Analysis}
\paragraph*{Chaos Game Representation}

\section{Results}

\section{Future}

\bibliographystyle{plain}
\bibliography{References}


\end{document}