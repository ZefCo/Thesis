\documentclass[12pt]{article}

\usepackage{amsmath}

\usepackage{graphicx}

\usepackage{hyperref}

\usepackage[utf8]{inputenc}

\title{Machine Learning with a Fractal Genome: working title}

\author{Ethan Speakman}

\date{2023–08–27}

\begin{document}

\maketitle

\section{Motivation & Introduction}
\paragraph*{}
The Spliceosome has one of the most important functions of the genome: the removal of introns from exons. 
This critcal step takes pre-mRNA and assembles exons in order for translation. 
A large ribonuclaticprotein strucutre, the Spliceosome will recuit several snRNPs which in turn bind to specific locations on the intron, fold the intron on itself, and splice it out, leaving only the exons\cite{SpliceStrucFunc}\cite[]{DayintheLife}.
For this reason a high fiedelity of intron and exon identification is required for the Spliceosome to properly exicise the introns: failure to do so could result tumerogensis down the line.
\paragraph*{}
Identification is made even more complicated by the role of alternative splicing.
Alternative splicing allows for a more diverse proteome as the roughly 20,000 genes of the human genome can be expressed in multiple ways: in humans exons can be skipped in different patterns, allowing a single gene to be expressed multiple ways.
In fact, any gene that has more then 3 exons can be alternativly expressed (citation).
This means that the Spliceosome needs not only to properly identify introns for removal, it also must identify which exons are to be skipped.
There are many factors that play into this,
\section{The Role of the Spliceosome}

\section{Background}
\section{Models}
\paragraph*{1D Time Analysis}
\paragraph*{Chaos Game Representation}

\section{Results}

\section{Future}


\end{document}